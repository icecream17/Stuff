\documentclass[]{article}
\usepackage{amssymb}
\usepackage{amsmath}
\usepackage{aligned-overset}
\usepackage{booktabs}% better tables
\usepackage[a4paper,
            bindingoffset=0.2in,
            left=1in,
            right=1in,
            top=0.5in,
            bottom=1in,
            footskip=.25in]{geometry}
\usepackage{changepage}
\usepackage{threeparttable}

\title{Outline of psdmul}
\author{Steven Nguyen (icecream17)}
\begin{document}

\maketitle

\noindent{}We have
\[
    (FG)(B) = \sum_{b \le B} F(b) G(B-b) \tag{mul}\label{mul}
\]
\[
    \frac{\partial F}{\partial X} = \left( k \mapsto \left(k(X) + 1\right) F(k+X) \right) \tag{psd}\label{psd}
\]
where $F$ and $G$ are multivariate polynomials (represented as functions from bags of variables to coefficients) and $X$ is a variable.

\noindent{}\hrulefill

\noindent{}We prove
\[
    \frac{\partial (FG)}{\partial X} = \left( \frac{\partial F}{\partial X} G \right) + \left( F \frac{\partial G}{\partial X} \right)
\]
from
\[
    \frac{\partial (FG)}{\partial X}(k) = \left( \frac{\partial F}{\partial X} G \right)\!\!(k) + \left( F \frac{\partial G}{\partial X} \right)\!\!(k) \tag{1 = 2 + 3}
\]
by expanding expressions 1, 2, and 3 as follows:

\vspace{-1em}
\begin{align*}
\frac{\partial (FG)}{\partial X}(k)  &= (k(X) + 1) \, (FG)(k+X) \\
&= (k(X) + 1) \sum_{b \le k+X} F(b) G((k+X)-b) \\
&= \sum_{b \le k+X} (k(X) + 1) F(b) G((k+X)-b)  \tag{1}\label{1}
\end{align*}
\hrulefill

\vspace{-1em}
\begin{align*}
\left( \frac{\partial F}{\partial X} G \right)\!\!(k)
= \sum_{b \le k}& \left( \frac{\partial F}{\partial X} \right)\!\!(b) G(k-b) \\
\underset{b = b' - X}{\longrightarrow} \sum_{X \le b' \le k+X}& \left( \frac{\partial F}{\partial X} \right)\!\!(b'-X) G(k-(b'-X)) \\
= \sum_{X \le b' \le k+X}& ((b'-X)(X) + 1) F((b'-X)+X) G(k-(b'-X)) \\
= \sum_{X \le b \le k+X}& b(X) F(b) G((k+X)-b)  \tag{2}\label{2}
\end{align*}
\hrulefill

\vspace{-1em}
\begin{align*}
\left( F \frac{\partial G}{\partial X} \right)\!\!(k)
&= \sum_{b \le k} F(b) \left( \frac{\partial G}{\partial X} \right)\!\!(k-b) \\
&= \sum_{b \le k} F(b) ((k-b)(X) + 1) G((k-b)+X) \\
&= \sum_{b \le k} ((k-b)(X) + 1) F(b) G((k-b)+X) \\
&= \sum_{b \le k} ((k-b)(X) + 1) F(b) G((k+X)-b)  \tag{3}\label{3}
\end{align*}
\hrulefill

Then we split each result, a sum, into summations over certain ranges, as shown in Table~\ref{table:sum}.

\newgeometry{left=0.1in,right=0.1in}
\begin{table}
\centering
\begin{threeparttable}
\caption{Final step: 2 + 3 = 1}\label{table:sum}
\begin{tabular}{lrrr}
    \toprule
    Reference & \multicolumn{1}{c}{$b < X$\tnote{*}} & \multicolumn{1}{c}{$X \le b \le k$} & \multicolumn{1}{c}{$k < b \le k+X$} \\
    \midrule
    \eqref{2} & & $b(X) F(b) G((k+X)-b)$ & $b(X) F(b) G((k+X)-b)$ \\
    \eqref{3} & $((k-b)(X) + 1) F(b) G((k+X)-b)$ & $((k-b)(X) + 1) F(b) G((k+X)-b)$ & \\
    \eqref{1} & $(k(X) + 1) F(b) G((k+X)-b)$ & $(k(X) + 1) F(b) G((k+X)-b)$ & $(k(X) + 1) F(b) G((k+X)-b)$ \\
    \bottomrule
\end{tabular}
\begin{tablenotes}
\item[*] bags without the variable $X$
\end{tablenotes}
\end{threeparttable}
\end{table}

\begin{adjustwidth}{3.9cm}{6.7cm}
For the first column, \[ b < X \]

means \[ b(X) = 0 \]

, so \[ (k-b)(X) = k(X) - b(X) = k(X) \]

.
\end{adjustwidth}

\begin{adjustwidth}{3.9cm}{6.7cm}
For the last column, \[ k < b \le k+X \]

implies \[ b(X) = (k+X)(X) = k(X) + 1 \]

.
\end{adjustwidth}

\end{document}
