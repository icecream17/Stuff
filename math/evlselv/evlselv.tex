\documentclass[]{article}
\usepackage{amssymb}
\usepackage{amsmath}
\usepackage{tikz-cd}
\usepackage{textcomp}

\usepackage{hyperref}
\hypersetup{
    colorlinks=true,
    linkcolor=blue,
    filecolor=magenta,
    urlcolor=blue
    }

\DeclareMathOperator{\eval}{eval}
\DeclareMathOperator{\select}{select}
\DeclareMathOperator{\polyVar}{polyVar}
\DeclareMathOperator{\lift}{lift}
\newcommand{\bag}[1]{\mathrm{bag}_{#1}}
\newcommand{\theoremlink}[1]{\href{https://us.metamath.org/mpeuni/#1.html}{\midtilde{} #1}}

% copied from https://tex.stackexchange.com/questions/312/correctly-typesetting-a-tilde
\makeatletter
\newcommand\midtilde@raisedtilde[1][.5]{\raisebox{#1ex}{\texttildelow}}
\def\midtilde@normaltilde{\texttildelow}
\newcommand\midtilde%
{%
  \expandafter\in@\expandafter{\f@family}%
    {cmr,cmss,cmtt,cmm,cmsy,cmx,%
    lmr,lmss,lmtt,lmm,lmsy,lmx,%
    pxr,pxss,pxm,pxsy,pxx,%
    txr,txss,txm,txsy,txx}%
  \ifin@%
    \midtilde@raisedtilde%
  \else%
    \expandafter\in@\expandafter{\f@family}%
    {pxtt,txtt}%
    \ifin@%
      \midtilde@raisedtilde[.35]%
    \else%
      \midtilde@normaltilde%
    \fi%
  \fi%
}

%opening
\title{Proving evlselv}
\author{Steven Nguyen (icecream17)}

\begin{document}

\maketitle

\begin{abstract}
Trying to prove that evaluating some variables, then the rest, is the same as evaluating all those variables at once, and succeeding after a few mistakes in explaining that I haven't cleaned up. Probably the hardest proof I've done.
\end{abstract}

\section{Introduction}

Facing a gap in understanding, I tried asking AIs to give me an overview of how one would prove this. The proof generally goes as follows: substitute some variables, and substitute the rest, tada!

The problem is, in metamath, the intermediate mathematical something that results when you only substitute some variables is very complicated.

\subsection{Definition rodeo}

A polynomial is a sum of several terms.

Each term is a coefficient times a "bag of variables" ($b \in \mathrm{bag}_v$).

A bag of variables maps variables in the set $V$ to their powers (in $\mathbb{N}_0$).

\indent{}\indent{}A bag $b \in \mathrm{bag}_V$ is a function $b: V \to \mathbb{N}_0$.

\noindent{}All in all, the polynomial ring in the set of variables $V$ over a ring $R$ is denoted $R[V]$, where a polynomial $p \in R[V]$ is a function from bags to coefficients: $p: \bag{V} \to R$

To evaluate a polynomial $p \in R[V]$, we need assignments from variables to values: a function $A: V \to R$. So, $\eval(P): (V \to R) \to R$, where

\[ \eval(P)(A) = \sum_{b_v \in \bag{V}}\left(P(b_v) \times \prod_{v \in V}{A(v)^{b_v(v)}}\right) \tag{ev}\label{ev} \]

This can be generalized to evaluation in a subring pretty trivially by \theoremlink{evlsevl}. Even more complicated is when it comes to selecting certain variables for evaluation. First, observe that we can lift $r \in R$ to a constant polynomial $p \in R[V]$.

% https://q.uiver.app/#q=WzAsNCxbMCwwLCJSIl0sWzIsMCwiUyJdLFswLDIsIlJbVl0iXSxbMiwyLCJTW1ZdIl0sWzAsMV0sWzAsMiwiXFxtYXRocm17bGlmdH0iLDAseyJzdHlsZSI6eyJ0YWlsIjp7Im5hbWUiOiJtb25vIn19fV0sWzEsMywiXFxtYXRocm17bGlmdH0iLDAseyJzdHlsZSI6eyJ0YWlsIjp7Im5hbWUiOiJtb25vIn19fV0sWzIsM11d
\[\begin{tikzcd}
	R && S \\
	\\
	{R[V]} && {S[V]}
	\arrow[from=1-1, to=1-3]
	\arrow["{\mathrm{lift}}", tail, from=1-1, to=3-1]
	\arrow["{\mathrm{lift}}", tail, from=1-3, to=3-3]
	\arrow[from=3-1, to=3-3]
\end{tikzcd}\]

So given a ring homomorphism from $R$ to $S$, we get the corresponding map from $R[V]$ to $S[V]$. In particular, the lift from $R$ to $R[V]$ is itself a ring homomorphism. Substituting $S$ with $R[V]$ we get a map from $R[V]$ to $R[V][W] = (R[V])[W]$. And we can chain infinitely: lifting $R$ to $R[V]$ to $R[V][W]$ and so on.

We can also lift a variable to the corresponding polynomial of a single term whose bag just has the variable and whose coefficient is one = 1. The polynomial of a variable $v$ will be denoted $\polyVar(v)$. This lift isn't really a homomorphism, but it can still be composed with homomorphisms.

So, here's how we select certain variables $J$ out of the index set of variables $I$ ($I$ is used more often instead of $V$). Let $p \in R[I]$. Selecting $J$ out will result in an element of $R[I-J][J]$, as follows:

\begin{multline*}
  \select(p)(J) =
  \sum_{b_i \in \bag{I}} \Bigg[(p(b_i) \in R \text{ lifted to } R[I-J][J]) \times \\
                          \left(\left(\prod_{k \in I-J}\polyVar(k)\right) \in R[I-J] \text{ lifted to } R[I-J][J]\right) \times \\
                           \prod_{j \in J}\polyVar(j)\Bigg]
\end{multline*}

...or, with creative use of scalar multiplication:

\[ \select(p)(J) = \sum_{b_i \in \bag{I}} \left[\left(p(b_i) \cdot
                                           \prod_{k \in I-J}\polyVar(k)\right) \cdot
                                           \prod_{j \in J}\polyVar(j)\right] \]

Finally, we have a concrete equation. The starting definition is actually more complicated since it actually lifts $p$ to $R[I-J][J][I]$ and evaluates that, but this is equivalent. Note that the assignments of a lifted polynomial $R[V][W]$ will map variables $W$ to $R[V]$ not $R$, so we can't just restrict the assignments $A$ but also lift them for the inner evaluation. Thus, given a polynomial $P \in R[I]$ and a set of assignments $A: I \to R$, we want to prove:

\[
  \eval(\eval(\select(P)(J))(\lift(A{\restriction_{J}})))(A{\restriction_{I-J}}) = \eval(P)(A) \tag{goal}\label{goal}
\]

where everything expands into oblivion.

\section{First steps}

Let's first look at the overall structure of \eqref{goal}. Upon expanding the evaluations using \eqref{ev} we get:

\[
  \sum_{b_k \in \bag{I-J}} \left[
     \left(\sum_{b_j \in \bag{J}} (
       \select(P)(J)(b_j)
     \times\dots)\right)(b_k)
  \times\dots\right] = \sum_{b_i \in \bag{I}} \left(p(b_i) \cdot \dots\right)
\]

where each ellipsis is the evaluation of the bag given the corresponding assignments.

An immediate issue is that the index sets of the summations are not equal. However, any $b_i \in \bag{I}$ (where $b_i: I\to\mathbb{N}_0$) can be decomposed into subset functions $b_j = b{\restriction_{J}}: J \to \mathbb{N}_0$ and $b_k = b{\restriction_{I-J}} : {I-J} \to \mathbb{N}_0$. As such the right hand side becomes

\begin{multline*}
  \eval(P)(A) = \sum_{b_i \in \bag{I}}\left(P(b_i) \times \prod_{i \in I}{A(i)^{b_i(i)}}\right) \\
              = \sum_{b_k \in \bag{I-J}} \sum_{b_j \in \bag{J}}
       \left(p(b_k \cup b_j) \times \prod_{i \in I}{A(i)^{(b_k \cup b_j)(i)}}\right) \\
              = \sum_{b_k \in \bag{I-J}} \sum_{b_j \in \bag{J}}
       \left(p(b_k \cup b_j) \times \prod_{k \in I-J}{A(k)^{b_k(k)}} \times \prod_{j \in J}{A(j)^{b_j(j)}}\right) \\
              = \sum_{b_k \in \bag{I-J}} \left[\prod_{k \in I-J}{A(k)^{b_k(k)}} \times \sum_{b_j \in \bag{J}}
       \left(p(b_k \cup b_j) \times \prod_{j \in J}{A(j)^{b_j(j)}}\right)\right]
\end{multline*}

We ignore the restrictions of $A$ since they don't affect the result. The outside product $\prod_{k \in I-J}{A(k)^{b_k(v)}}$ corresponds to the outside ellipsis on the left hand side! So now our goal is:

\begin{multline} \label{goal2}
  \sum_{b_k \in \bag{I-J}} \left[\left(\sum_{b_j \in \bag{J}} \left(
       \select(P)(J)(b_j)
     \times \prod_{j \in J}{\lift\left(A(j)\right)^{b_j(j)}}\right) \right)(b_k)\right] \\
  = \sum_{b_k \in \bag{I-J}} \sum_{b_j \in \bag{J}}
       \left(p(b_k \cup b_j) \times \prod_{j \in J}{A(j)^{b_j(j)}}\right)
\end{multline}

\section{Dozens of steps in already}

It doesn't seem like that last equation can be manipulated much more, so we'll have to expand $\select(P)(J)(b_j)$.

\begin{equation*}
\begin{split}
  \select(P)(J)(b_j) \\ = \left(\sum_{b_i \in \bag{I}} \left[\left(P(b_i) \cdot
                                           \prod_{k \in I-J}\polyVar(k)\right) \cdot
                                           \prod_{j \in J}\polyVar(j)\right]\right)(b_j)
\end{split}
\end{equation*}

By \theoremlink{gsummhm}, if we apply a (monoid) homomorphism to every addend of a sum, it is the same as applying the homomorphism to the whole sum. So by showing \[ ( p \in R[I-J][J] \mapsto p(b_j) ) \] is a monoid homomorphism, we can move the function application $(\sum \dots)(b_j)$ inside: $\sum (\dots (b_j))$. Since a product sums over the multiplicative group, and a ring homomorphism provides a monoid homomorphism over the multiplicative groups (by \theoremlink{rhmmhm}), we can also do this with products. The proof that the function above is a ring homomorphism is hand-waved by referencing \theoremlink{evls1maprhm}.

So clearly, equation \ref{goal2} can be manipulated after all. Firstly,

\[
  \prod_{j \in J}{\lift\left(A(j)\right)^{b_j(j)}}
  = \prod_{j \in J}\lift\left(A(j)^{b_j(j)}\right)
  = \lift\left(\prod_{j \in J}A(j)^{b_j(j)}\right)
\]

So the left hand side becomes:
\begin{align*}
    &\sum_{b_k \in \bag{I-J}} \left[\left(\sum_{b_j \in \bag{J}} \left(
       \select(P)(J)(b_j)
     \times \prod_{j \in J}{\lift\left(A(j)\right)^{b_j(j)}}\right) \right)(b_k)\right] \\
  = &\sum_{b_k \in \bag{I-J}} \sum_{b_j \in \bag{J}} \left[\left(
       \select(P)(J)(b_j)
     \times \prod_{j \in J}{\lift\left(A(j)\right)^{b_j(j)}} \right)(b_k)\right] \\
  = &\sum_{b_k \in \bag{I-J}} \sum_{b_j \in \bag{J}} \left[\left(
       \select(P)(J)(b_j)
     \times \lift\left(\prod_{j \in J}A(j)^{b_j(j)}\right) \right)(b_k)\right] \\
  = &\sum_{b_k \in \bag{I-J}} \sum_{b_j \in \bag{J}} \left(
       \select(P)(J)(b_j)(b_k)
     \times \lift\left(\prod_{j \in J}A(j)^{b_j(j)}\right)(b_k) \right) \\
\end{align*}

The last part is a property of being a ring homomorphism. And now, equating that last expression with the right hand side, if we can show:

\begin{align*}
  \select(P)(J)(b_j)(b_k) &= p(b_k \cup b_j) \text{, and}\\
  \lift\left(\prod_{j \in J}A(j)^{b_j(j)}\right)(b_k) &= \prod_{j \in J}{A(j)^{b_j(j)}}
\end{align*}

then we have done it!!

\section{Library of Alexandria}

This section title is not because valuable information is lost but because I feel on fire finally figuring it out.

\[ \lift\left(\prod_{j \in J}A(j)^{b_j(j)}\right)(b_k) = \prod_{j \in J}{A(j)^{b_j(j)}} \]

Ok, the second equation isn't directly true for all $b_k$, but it is true when we sum over $b_k$. A lifted constant only has a nonzero coefficient for the bag of all variables raised to the power of zero, so setting $b_k$ to that bag, we get the result.

Similarly, for the first equation, $\lift(P(b_i))$ is a lifted constant, so its coefficient is only nonzero when setting $b_j$ and $b_k$ to identity-1-bags. So we get:

\begin{align*}
     &\select(P)(J)(b_j)(b_k)
  \\ &= \left(\sum_{b_i \in \bag{I}} \left[\left(P(b_i) \cdot
           \prod_{k \in I-J}\polyVar(k)\right) \cdot
           \prod_{j \in J}\polyVar(j)\right]\right)(b_j)(b_k)
  \\ &= \sum_{b_i \in \bag{I}} \left[\lift(P(b_i))(b_j)(b_k) \times \dots \right]
  \\ &= \sum_{b_i \in \bag{I}} \left[\lift(P(b_i))(1)(1) \times \dots \right]
  \\ &= \sum_{b_i \in \bag{I}} P(b_i) \times
           \lift\left(\prod_{k \in I-J}\polyVar(k)\right)(1)(1) \times
           \prod_{j \in J}\polyVar(j)(1)(1)
   \\&= \sum_{b_i \in \bag{I}} P(b_i) \times 1 \times 1
   \\&= \sum_{b_k \in \bag{I-J}} \sum_{b_j \in \bag{J}} p(b_k \cup b_j)
\end{align*}

which was what we wanted. (Note: variables have coefficient 1)

\end{document}
